\documentclass[11pt]{article}

% \usepackage[
% backend=biber,
% style=alphabetic,
% sorting=ynt
% ]{biblatex}

\usepackage[square,numbers]{natbib}

\bibliographystyle{plainnat}

% \addbibresource{references.bib}

\usepackage{fullpage}

\topmargin=-0.2in
\textheight=9.5in

\title{Term Project Proposal}
\author{Insoo Chung (UIN: 232004620)}

\begin{document}
\maketitle
\section{Project Outline}

To enable runtime countermeasures for adversarial attacks, \citet{ptolemy} recently proposed a hardware module that detects adversarial attacks with very low overhead.
The proposed module was able to achieve high detection accuracy while maintaining a very low detection latency - 2\% runtime overhead as compared to the previous methods.
Such feat of efficiency could be achieved by focusing on distinctive paths that adversarial inputs activate.

The aim of this term project is to recreate the detector and assess its efficacy.
The efficiency of the detector will be measured in terms of area and power overhead of the module implementation.

\section{Background}

In the context of neural networks, \textit{adversarial attacks} refer to attacks on deployed machine learning (ML) models with intention to cause failure.
The resulting failures from these attacks may have fatal results.
For example, for autonomous driving, a successful attack may result in severe injuries or even death.
\citet{szegedy2013intriguing} found that adversarial inputs can be devised to resemble a perfectly well-formed inputs while still causing neural networks to fail.
Thus, such attacks may occur in the most discrete forms and could be very hard to prevent.
While various countermeasures have been studied \cite{detected,ensembles}, the studied methods require high computation from fully examining the inputs, and cannot be applied to to be deployed for important realtime ML applications (e.g. again, autonomous driving).

\section{Expected Output}

This project aims to provide implementation of relevant modules, enhanced MAC unit and sorting logic.
Also, efficiency of the proposed module is to be measured in terms of area and power overhead.

\begin{itemize}
    \item Implemenations:
    \begin{itemize}
        \item Enhanced MAC unit in verilog.
        \item Sorting logic in verilog with timing and energy estimation using Synergy's synthesis.
    \end{itemize}
    \item Efficiency evaluation:
    \begin{itemize}
        \item Power and area overhead of the detector will be measured to evaluate the module's efficiency.
    \end{itemize}
\end{itemize}

\section{Approach and Milestones}

\begin{itemize}
    \item Write proposal (Week 9)
    \item Implement enhanced MAC unit. (Week 10-11)
    \item Implement testing logic around the enhanced MAC unit. (Week 10-11)
    \item Implement sorting logic. (Week 11)
    \item Setup Synopsis synthesis for evaluation of the developed sorting logic. (Week 11-13)
    \item Perform evaluation. (Week 12-13)
    \item Write final report (Week 9-14)
\end{itemize}


\bibliography{references}
\end{document}